\documentclass[12pt,a4paper]{article}
\usepackage[utf8]{inputenc}
\usepackage[portuguese]{babel}
\usepackage{graphicx, hyperref, verbatim, multicol, amsmath, listings}

\author{Grupo XX \and Daniel Fernandes 86400 \& \\Francisco Sousa 86416 \& Henrique Ferreira 86432}
\title{Projeto de Bases de Dados, Parte 2}
\begin{document}
\maketitle
\newpage

\section{Introdução}
\section{Modelo Relacional}

EntidadeMeio(\underline{nomeEntidade}) \\\\
Coordenador(\underline{idCoordenador}) \\\\
Local(\underline{moradaLocal}) \\\\
Camara(\underline{numCamara}) \\\\
Meio(\underline{nomeEntidade}, \underline{numMeio}, nomeMeio)

nomeEntidade: FK(EntidadeMeio) \\\\
MeioCombate(\underline{nomeEntidade}, \underline{numMeio}, \underline{NomeMeio})

nomeEntidade: FK(Meio)

NumMeio: FK(Meio)

nomeMeio: FK(Meio) \\\\
MeioApoio(\underline{nomeEntidade}, \underline{numMeio}, \underline{NomeMeio})

nomeEntidade: FK(Meio)

NumMeio: FK(Meio)

nomeMeio: FK(Meio) \\\\
MeioSocorro(\underline{nomeEntidade}, \underline{numMeio}, \underline{NomeMeio})

nomeEntidade: FK(Meio)

numMeio: FK(Meio)

nomeMeio: FK(Meio) \\\\
ProcessoSocorro(\underline{numProcessoSocorro}) \\\\
EventoEmergência(\underline{numTelefone}, \underline{instanteChamada}, nomePessoa) \\\\
segmentoVideo(\underline{numCamera}, \underline{dataHoraInicio}, \underline{numSegmento}, duração)

numCamara: FK(video)

dataHoraInicio: FK(video) \\\\
Video(\underline{numCamara}, \underline{dataHoraInicio}, dataHoraFim)

numCamara: FK(video) \\\\
Transporta(\underline{numProcessoSocorro}, \underline{nomeEntidade}, \underline{numMeio}, numVitimas )

numProcessoSocorro: FK(ProcessoSocorro)

nomeEntidade: FK(Meio)

numMeio: FK(Meio) \\\\
Alocado(\underline{numProcessoSocorro}, \underline{numMeio}, numHoras)

NumProcessoSocorro: FK(processoSocorro) \\\\
localIncendio(\underline{numTelefone}, \underline{instanteChamada}, \underline{moradaLocal})

instanteChamada: FK(EventoEmergência)

numTelefone: FK(EventoEmergência)

moradaLocal: FK(Local) \\\\
Origina(\underline{numTelefone}, \underline{instanteChamada}, \underline{numProcessoSocorro})  ?

numTelefone: FK(EventoEmergência)

instanteChamada: FK(EventoEmergência)

numProcessoSocorro: FK(ProcessoSocorro) \\\\
acciona(\underline{numProcessoSocorro}, \underline{numMeio},  \underline{nomeEntidade})

numProcessoSocorro: FK(ProcessoSocorro)

numMeio: FK(Meio)

nomeEntidade: FK(Meio) \\\\
audita(\underline{numProcessoSocorro}, \underline{idCoordenador}, \underline{numMeio}, \underline{nomeEntidade}, datahoraInicio, texto, datahorafim, dataAuditoria)

NumProcessoSocorro, numMeio, nomeEntidade: FK(acciona)

idCoordenador: FK(Coordenador) \\\\
Vigia(\underline{moradaLocal}, \underline{numCamara})

moradaLocal: FK(Local)

numCamara: FK(Camara) \\\\
Solicita(\underline{idCoordenador}, \underline{dataHoraInicio}, \underline{numCamara}, datahorainicio, datahorafim)

idCoordenador: FK(Coordenador)

dataHoraInicio: FK(Vídeo) 

numCamara: FK(Camara) \\\\

\section{Restrições Referentes ao Modelo Relacional}


\textbf{RI1} - Quando o EntidadeMeio for eliminada os correspondentes Meios também são.
\textbf{RI2} - Qualquer Evento de Emergência origina no máximo um ProcessoSocorro
\textbf{RI3} - Um EventoEmergência tem um e um só Local.
\textbf{RI4} - Um ProcessoSocorro tem de ser originado por pelo menos Evento de Emergência
\textbf{RI5} - Quando o Vídeo for eliminado os correspondentes segmentos de vídeos também são.
\textbf{RI6} - Quando camara for eliminada os correspondentes vídeos também são.


\section{Restrições comuns entre modelo E-A e Modelo Relacional}
\textbf{RI7} - O coordenador só pode solicitar vídeos de períodos temporais que tenha auditado;
\textbf{RI8} - A data-hora de fim da auditoria tem de ser posterior à data-hora de inicio;
\textbf{RI9} - A data da auditoria tem de ser anterior ou igual ao momento atual;
\textbf{RI10} - Um meio de socorro apenas pode transportar vítimas de processos de socorro onde
tenha sido acionado;
\textbf{RI11} - Um meio de apoio apenas pode ser alocado a processos de socorro onde tenha sido acionado
\textbf{RI12} - Conjuntamente, os atributos “numTelefone” e “nomePessoa”, podem ser usados para
identificar um evento de emergência;
\textbf{RI13} - O somatório do número de segmentos de um vídeo multiplicados pelas suas res
durações deve ser igual à diferença entre a data-hora de fim e de início do vídeo;
\textbf{RI14} - Para um determinado ProcessoSocorro, um meio não pode simultaneamente
fornecedor Meio de Apoio e Meio de Combate.
\textbf{RI15} - Qualquer ProcessoSocorro origina EventoEmergencia \\

\textbf{RI16} - meio pertence a uma e s\'o uma entidade \\


\section{Álgebra Relacional}
\paragraph{1}
Liste todos os meios de Socorro (número do Meio e entidade proprietária) que foram
usados em incêndios cujos eventos foram registados em “Palmela” ou “Moita”, entre
10/8/2018 às 00:00 e 14/8/2018 às 23:59;

$filtradoInstante \leftarrow \sigma_{10/8/2018\,00:00\, < \, instanteChamada\,  < \, 14/8/2018 \, 23:59} (localIncendio \bowtie origina)$

$\Pi_{nomeEntidade}\left ( Transporta \bowtie \left (  \sigma_{moradaLocal=Pamela\, \vee \, moradaLocal=Moita } \left ( filtradoInstante \right )\right ) \right )$

\paragraph{2}
Liste os locais em que um mesmo número de telefone foi usado 2 ou mais vezes para
reportar eventos de emergência;

$\Pi_{nomeEntidade, nomeMeio} \left ( Transporta \bowtie \left ( \sigma_{count\geq 2}  \left (_{numTelefone}G_{count()}  \left (localIncendio \right ) \right ) \right ) \right )$

\paragraph{3}
Qual é o processo de socorro que envolveu maior número de meios distintos;

$difnumMeio \leftarrow\left ( _{numProcessoSocorro}G_{count()}acciona \right )$

$Max \leftarrow G_{max(count)} \left ( difnomeMeio \right )$

$\Pi_{umProcessoSocorro}\left ( \sigma _{max=count}\left ( Max\times difnumMeio \right ) \right )$

\paragraph{4}
Qual a entidade fornecedora de meios que participou em mais processos de socorro no
Verão de 2018;

$filtradoInstante \leftarrow \sigma_{21/6/2018 00:00 < instanteChamada < 22/9/2018 23:59} \left ( acciona \bowtie origina \right ) $

$NrEntidade \leftarrow\left ( _{nomeEntidade}G_{count()} filtradoInstante \right )$

$Max \leftarrow G_{max(count)} \left ( NrEntidade \right )$

$\Pi_{nomeEntidade}\left ( \sigma _{max=count}\left ( Max\times NrEntidade \right ) \right )$

\paragraph{5}
Quais são os processos de socorro, referente a eventos de emergência em 2018 de
Oliveira do Hospital, onde existe pelo menos um acionamento de meios que não foi alvo
de auditoria;

$N \leftarrow localIncendio \bowtie \left( origina - origina \bowtie audita \right)$

$\Pi_{numProcessoSocorro} \left( \sigma_{moradaLocal=Oliveira do Hospital \, \wedge \, 1/1/2018 00:00 < instanteChamada < 31/12/2018 23:59} \left(N\right)\right) $

\paragraph{6}
Quantos segmentos de vídeo com duração superior a 60 segundos, foram gravados em
câmeras de vigilância de Monchique durante o mês de Agosto de 2018;

$group \leftarrow vigia \bowtie video \bowtie segmentoVideo$

$F \leftarrow \sigma_{duracao > 60 \wedge moradaLocal = Monchique \wedge dataHoraInicio > 1/8/2018 00:00 \wedge dataHoraFim < 31/8/2018 23:59} \left( group\right) $

$\Pi_{count} \left(G_{count()} \left( F \right) \right) $

\paragraph{7}
Liste os Meios de combate que não foram usados como Meios de Apoio em nenhum
processo de socorro;

$\Pi \left( MeioCombate\right) - \Pi \left( MeioCombate \bowtie \left(MeioApoio \bowtie processoSocorro\right)\right) $

\paragraph{8}
Lista as entidades que forneceram meios de combate a todos os Processos de socorro
que acionaram meios;

$\Pi_{EntidadeMeio} \left(acciona\right) \div \Pi \left(MeioCombate\right)$

\section{SQL}

\paragraph{1}
\begin{lstlisting}[
    breaklines, 
    language=SQL,
 ]
select nomeEntidade from Transporta natural join (select * from (select * from localIncendio natural join origina where 10/8/2018 00:00 < instanteChamada < 14/8/2018 23:59) where moradaLocal = Palmela or moradaLocal = Moita)
\end{lstlisting}

\paragraph{2}
\begin{lstlisting}[
    breaklines, 
    language=SQL,
 ]
select nomeEntidade, nomeMeio from Transporta natural join (select * from (select count(numTelefone) from localIncendio) where count >=2)
\end{lstlisting}

\end{document} 
