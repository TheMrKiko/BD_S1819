\documentclass[10pt,a4paper]{article}
\usepackage[utf8]{inputenc}
\usepackage[portuguese]{babel}
\usepackage{graphicx, hyperref, verbatim, multicol, amsmath, listings}

\addtolength{\oddsidemargin}{-1.4in}
\addtolength{\evensidemargin}{-1.4in}
\addtolength{\textwidth}{2.8in}
\addtolength{\topmargin}{-.875in}
\addtolength{\textheight}{1.75in}

\author{Grupo 47 \\\\ Daniel Fernandes 86400 \& \\Francisco Sousa 86416 \& Henrique Ferreira 86432}
\title{Projeto de Bases de Dados, Parte 3}
\begin{document}

\maketitle

\begin{center}
Turno BD22517957L09 \\
Sexta 12:30 Lab 8 \\
Prof. Taras Lykhenko \\
\end{center}

\begin{table}[h]
    \centering
    \begin{tabular}{lll}
    \hline
    \textbf{Número de Aluno} & \textbf{Nome} & \textbf{Esforço} \\ \hline
    86400 & Daniel Fernandes & 33\% (20h) \\ \hline
    86416 & Francisco Sousa & 33\% (20h) \\ \hline
    86432 & Henrique Ferreira & 33\% (20h) \\ \hline
    \end{tabular}
\end{table}
\newpage

\section{Criação e População da Base de Dados}

Para a criação da base de dados foi usado o código em seguida apresentado.
Em primeiro lugar é limpa a base de dados e em seguida são criadas as tabelas de acordo com o modelo relacional do enunciado. 

\begin{verbatim}
drop table solicita cascade;
drop table audita cascade;
drop table coordenador cascade;
drop table acciona cascade;
drop table alocado cascade;
drop table transporta cascade;
drop table meio cascade;
drop table meio_socorro cascade;
drop table meio_apoio cascade;
drop table meio_combate cascade;
drop table entidade_meio cascade;
drop table evento_emergencia cascade;
drop table processo_socorro cascade;
drop table vigia cascade;
drop table localidade cascade;
drop table camara cascade;
drop table video cascade;
drop table segmento_video cascade;

create table camara 
    (num_camara char(5) not null unique,
     constraint pk_camara primary key(num_camara));

create table video
    (data_hora_inicio timestamp not null,
     data_hora_fim    timestamp not null,
     num_camara       char(5)   not null,
     constraint pk_video primary key(data_hora_inicio, num_camara),
     constraint fk_num_camara foreign key(num_camara) references camara(num_camara),
     constraint ck_data_hora check (data_hora_inicio < data_hora_fim));

create table segmento_video
    (num_segmento     char(5)    not null,
     duracao          int not null,
     data_hora_inicio timestamp  not null,
     num_camara       char(5)    not null,
     constraint pk_segmento_video primary key(num_segmento, data_hora_inicio, num_camara),
     constraint fk_segmento_video foreign key(data_hora_inicio, num_camara) 
        references video(data_hora_inicio, num_camara));

create table localidade
    (morada_local varchar(255) not null,
    constraint pk_morada_local primary key(morada_local));

create table vigia
    (morada_local varchar(255) not null,
     num_camara   char(5) not null unique,
     constraint pk_vigia primary key(morada_local, num_camara),
     constraint fk_morada_local foreign key(morada_local) references localidade(morada_local),
     constraint fk_num_camara foreign key(num_camara) references camara(num_camara));

create table processo_socorro
    (num_processo_socorro char(5) not null unique,
     constraint pk_num_processo_socorro primary key(num_processo_socorro));

create table evento_emergencia
    (num_telefone         char(9) not null,
     instante_chamada     timestamp not null,
     nome_pessoa          varchar(255) not null,
     morada_local         varchar(255) not null,
     num_processo_socorro char(5) not null,
     constraint pk_evento_emergencia primary key(num_telefone, instante_chamada),
     constraint fk_morada_local foreign key(morada_local) references localidade(morada_local),
     constraint fk_num_processo_socorro foreign key(num_processo_socorro) 
        references processo_socorro(num_processo_socorro));

create table entidade_meio
    (nome_entidade varchar(20) not null unique,
     constraint pk_nome_entidade primary key(nome_entidade));

create table meio
    (num_meio      char(5)     not null,
     nome_meio     varchar(255) not null,
     nome_entidade varchar(20) not null,
     constraint pk_meio primary key(num_meio, nome_entidade),
     constraint fk_nome_entidade foreign key(nome_entidade) references entidade_meio(nome_entidade));

create table meio_combate
    (num_meio      char(5)     not null,
     nome_entidade varchar(255) not null,
     constraint pk_meio_combate primary key(num_meio, nome_entidade),
     constraint fk_meio foreign key(num_meio, nome_entidade) references meio(num_meio, nome_entidade));

create table meio_apoio
    (num_meio       char(5)    not null,
     nome_entidade  varchar(255) not null,
     constraint pk_apoio primary key(nome_entidade, num_meio),
     constraint fk_meio foreign key(num_meio, nome_entidade) references meio(num_meio, nome_entidade));

create table meio_socorro
    (num_meio       char(5)      not null,
     nome_entidade  varchar(255) not null,
     constraint pk_socorro primary key(nome_entidade, num_meio),
     constraint fk_meio foreign key(num_meio, nome_entidade) references meio(num_meio, nome_entidade));

create table transporta
    (num_meio             char(5)    not null,
     nome_entidade        varchar(20) not null,
     num_vitimas          int,
     num_processo_socorro char(5),
     constraint pk_transporta primary key(num_processo_socorro, nome_entidade, num_meio),
     constraint fk_transporta_num_meio foreign key(num_meio, nome_entidade)
        references meio_socorro(num_meio, nome_entidade),
     constraint fk_transporta_num_proc foreign key(num_processo_socorro)
        references processo_socorro(num_processo_socorro));
    
create table alocado
    (num_meio             char(5)    not null,
     nome_entidade        varchar(20) not null,     
     num_horas            decimal(5, 2),
     num_processo_socorro char(5),
     constraint pk_alocado primary key(num_processo_socorro, nome_entidade, num_meio),
     constraint fk_alocado_num_meio foreign key(num_meio, nome_entidade) 
        references meio_apoio(num_meio, nome_entidade),
     constraint fk_alocado_num_proc foreign key(num_processo_socorro) 
        references processo_socorro(num_processo_socorro));

create table acciona
    (num_meio             char(5)    not null,
     nome_entidade        varchar(20) not null,
     num_processo_socorro char(5),
     constraint pk_acciona primary key(num_processo_socorro, nome_entidade, num_meio),
     constraint fk_acciona_num_meio foreign key(num_meio, nome_entidade)
        references meio(num_meio, nome_entidade),
     constraint fk_acciona_num_proc foreign key(num_processo_socorro) 
        references processo_socorro(num_processo_socorro));

create table coordenador
    (id_coordenador char(5) not null,
     constraint pk_coordenador primary key(id_coordenador));

create table audita
    (id_coordenador       char(5)    not null,
     num_meio             char(5)    not null,
     nome_entidade        varchar(20) not null,
     num_processo_socorro char(5),
     data_hora_inicio     timestamp,
     data_hora_fim        timestamp,
     data_auditoria       date,
     texto                varchar(100),
     constraint pk_audita primary key(id_coordenador, num_meio, nome_entidade, num_processo_socorro),
     constraint fk_audita_num_meio foreign key(num_meio, nome_entidade, num_processo_socorro)
        references acciona(num_meio, nome_entidade, num_processo_socorro),
     constraint fk_audita_id_coord foreign key(id_coordenador) references coordenador(id_coordenador),
     constraint ck_data_autoria check (data_auditoria <= now()));

create table solicita
    (id_coordenador         char(5)   not null,
     data_hora_inicio_video timestamp not null,
     num_camara             char(5)   not null unique,
     data_hora_inicio       timestamp not null,
     data_hora_fim          timestamp not null,
     constraint pk_solicita primary key(id_coordenador, data_hora_inicio_video, num_camara),
     constraint fk_solicita_id_coord     foreign key(id_coordenador) references coordenador(id_coordenador),
     constraint fk_solicita_inicio_video foreign key(data_hora_inicio_video, num_camara)
        references video(data_hora_inicio, num_camara));

\end{verbatim}

Para a população da base de dados foi criado um script em Python que cria o populate.sql com atenção às restrições de integridade. 

\section{SQL}

\paragraph{1.}
\begin{verbatim}
select * from (select num_processo_socorro, count(*) from acciona 
group by num_processo_socorro) as c cross join 
(select max(count) from (select num_processo_socorro, count(*) from acciona 
group by num_processo_socorro) as b) as a where (count = max);
\end{verbatim}

\paragraph{2.}
\begin{verbatim}
select  * from (select * from (select nome_entidade, count(*) from 
(select * from (acciona natural join evento_emergencia) as b 
where instante_chamada > '2018-06-21 00:00:00' and instante_chamada < '2018-09-22 23:59:59')
as c group by nome_entidade) as b cross join 
(select max(count) from (select nome_entidade, count(*) from 
(select * from (acciona natural join evento_emergencia) as b where instante_chamada > '2018-06-21 00:00:00' 
and instante_chamada < '2018-09-22 23:59:59') as c group by nome_entidade) as d) as e) as ahhhhhh
where count = max;

\end{verbatim}

\paragraph{3.}
\begin{verbatim}
select num_processo_socorro from (evento_emergencia natural join 
(select num_processo_socorro from acciona except 
select num_processo_socorro from (acciona natural join audita)) as b) 
where (morada_local = 'Oliveira do Hospital' 
and instante_chamada > '2018-01-01 00:00:00' and instante_chamada < '2018-12-31 23:59:59');
\end{verbatim}

\paragraph{4.}
\begin{verbatim}
select count(*) from 
(select * from (vigia natural join segmento_video natural join video) as b 
where (duracao > 60 and morada_local = 'Monchique' and
data_hora_inicio > '2018-08-01 00:00:00' and data_hora_fim < '2018-08-31 23:59:59')) as a;
\end{verbatim}

\paragraph{5.}
\begin{verbatim}
select nome_entidade from (meio_combate natural join 
(select num_meio from meio_combate except select num_meio from  
(select num_meio from acciona except select num_meio from 
(select num_meio from acciona except select num_meio from meio_apoio) as b) as c) as marrucho);
\end{verbatim}

\paragraph{6.}
\begin{verbatim}
select nome_entidade from meio_combate where not exists 
(select num_processo_socorro from processo_socorro except 
select num_processo_socorro from (acciona natural join processo_socorro));
\end{verbatim}

\section{Desenvolvimento da Aplicação}
A "entrada" da nossa aplicação é um ficheiro \textbf{index.html} que têm links para tudo o resto,
separados por duas seções:

\subsection{Ver}
Tem todas as tabelas listadas, permitindo consultá-las na íntegra.
Se tiver sido pedido, ainda é possível \textit{adicionar}, \textit{editar} ou \textit{remover}
entradas de cada tabela. 

A consulta é conseguida com o ficheiro \textbf{list.php} que, por sua vez, constrói os links
para as operações:

\subsubsection{Adicionar} Pede os valores necessários ao utilizador (\textbf{addform.php}) e adiciona (\textbf{add.php});
\subsubsection{Editar} Mostra todos os campos alteráveis ao utilizador (\textbf{editform.php}) e edita o que este alterou (\textbf{edit.php});
\subsubsection{Remove} Apaga todas as entradas necessárias de cada tabela, para manter as restrições (\textbf{remove.php}).

\subsection{Consultar}
Tem as duas consultas pedidas no enunciado. Para cada uma delas, pergunta ao utilizador o valor
do campo a procurar (\textbf{consultform.php} e apresenta a consulta (\textbf{consult.php}).




\end{document} 