\documentclass[10pt,a4paper]{article}
\usepackage[utf8]{inputenc}
\usepackage[portuguese]{babel}
\usepackage{graphicx, hyperref, verbatim, multicol, amsmath, listings}

\addtolength{\oddsidemargin}{-1.4in}
\addtolength{\evensidemargin}{-1.4in}
\addtolength{\textwidth}{2.8in}
\addtolength{\topmargin}{-.875in}
\addtolength{\textheight}{1.75in}

\author{Grupo 47 \\\\ Daniel Fernandes 86400 \& \\Francisco Sousa 86416 \& Henrique Ferreira 86432}
\title{Projeto de Bases de Dados, Parte 4}
\begin{document}

\maketitle

\begin{center}
Turno BD22517957L09 \\
Sexta 12:30 Lab 8 \\
Prof. Taras Lykhenko \\
\end{center}

\begin{table}[h]
    \centering
    \begin{tabular}{lll}
    \hline
    \textbf{Número de Aluno} & \textbf{Nome} & \textbf{Esforço} \\ \hline
    86400 & Daniel Fernandes & 33\% (10h) \\ \hline
    86416 & Francisco Sousa & 33\% (10h) \\ \hline
    86432 & Henrique Ferreira & 33\% (10h) \\ \hline
    \end{tabular}
\end{table}
\newpage

\section{Restrições de Integridade}

Os mecanismos que achamos mais apropriados para a definição destas restrições
de integridade são os \textbf{triggers}. Fazem consultas mais complexas do que o \textbf{CHECK}

\paragraph{a)}
\begin{verbatim}
CREATE OR REPLACE FUNCTION chk_coord_local_on_solic_proc() 
RETURNS TRIGGER
AS $$
DECLARE morada_camara VARCHAR;
BEGIN
    SELECT morada_local INTO morada_camara FROM vigia WHERE num_camara = NEW.num_camara;
    IF EXISTS 
        (SELECT * FROM (audita NATURAL JOIN evento_emergencia) AS r
            WHERE r.morada_local = morada_camara
            AND r.id_coordenador = NEW.id_coordenador)
    THEN
        RETURN NEW;
    ELSE
        RAISE EXCEPTION 'Um coordenador so pode solicitar videos de camaras colocadas num local
        cujo accionamento de meios esteja a ser (ou tenha sido) auditado por ele proprio!';
        RETURN NULL;
    END IF;
END;
$$ LANGUAGE plpgsql;

DROP trigger IF EXISTS chk_coord_local_on_solic ON solicita;
CREATE trigger chk_coord_local_on_solic BEFORE INSERT ON solicita for each ROW EXECUTE procedure chk_coord_local_on_solic_proc();
    
\end{verbatim}

\paragraph{b)}
\begin{verbatim}
CREATE OR REPLACE FUNCTION chk_acciona_on_alocado_proc()
RETURNS TRIGGER
AS $$
BEGIN
    IF EXISTS
        (SELECT * FROM acciona
            WHERE num_meio = NEW.num_meio
            AND nome_entidade = NEW.nome_entidade
            AND num_processo_socorro = NEw.num_processo_socorro)
    THEN
        RETURN NEW;
    ELSE
        RAISE EXCEPTION 'Um Meio de Apoio so pode ser alocado a Processos de Socorro para
        os quais tenha sido accionado!';
        RETURN NULL;
    END IF;
END;
$$ LANGUAGE plpgsql;

DROP trigger IF EXISTS chk_acciona_on_alocado ON alocado;
CREATE trigger chk_acciona_on_alocado BEFORE INSERT ON alocado for each ROW EXECUTE procedure chk_acciona_on_alocado_proc();
    
\end{verbatim}
\section{Índices}

\paragraph{a)}
Para a primeira interrogação, justificaria ter índices do tipo \_ porque \_, para os atributos \_ das tabelas \_.

Para a segunda interrogação, justificaria ter índices do tipo \_ porque \_, para os atributos \_ das tabelas \_.
\paragraph{b)}
Os índices que propomos são:
\begin{verbatim}
\end{verbatim}

\section{Modelo Multidimensional}
\begin{verbatim}
\end{verbatim}


\section{Data analytics}
\begin{verbatim}
\end{verbatim}



\end{document} 